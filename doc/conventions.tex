%!xelatex
\documentclass[main.tex]{subfiles}
\begin{document}

\chapter{Conventions}\label{chp:conventions}

\section{User Assumptions}

You, the user of @CC, are assumed to understand American crochet terminology, and be competent in the basic usage of the device that is running @CC.

\section{Computer System Assumptions}\label{sec:csassumptions}

@CC runs in a web browser, so it is assumed that you are using a browser that is up-to-date, and has not been tampered with.
It is assumed that you are using a device with a keyboard and mouse.
If you wish to use any features of the computer filesystem, it is assumed that your device has sufficient free space for these features.

\section{Notational Conventions}

The following text conventions are used in this manual:

\begin{itemize}
\item Libertine is used for normal text.
\item \emph{Libertine italicized} is used for new terms in normal text.
\item \textbf{Libertine bolded} is used for emphasis.
% \item Libertine in gray is used for section titles. For example, the title of the ``Conventions'' section is in Heading 3.
\item \io{Noto Sans} is used for program names and user interface input/output.
\item \ui{Noto Sans italicized} is used for widget names.
\item \Verb`Bitstream Vera Sans Mono` is used for @ir contents.
\item \emph{\Verb`Bitstream Vera Sans Mono italicized`} is used for variables of @ir syntax.
\end{itemize}

\section{Terms}\label{sec:terms}

Listing~\ref{lst:grammar} details the basic \term{grammar} used when parsing the pattern text.
Furthermore, table~\ref{tbl:keywords} contains the list of \term[grammar!keywords]{keywords}.
In each row, the keywords listed next to each other can be used inter\-changeably.
For example, \io{ss} has the same meaning as \io{slst}, so whenever the grammar includes ``\io{slst},'' \io{ss} can also be used.

\begin{listing}[htbp]
    \begin{tcolorbox}[breakable, parbox = false, colback = black!5, colframe = black!30]
        {\newcommand{\nonterminal}[1]{\emph{\Verb`#1`}}
\newcommand{\terminal}[1]{``\io{#1}''}
\newcommand{\defeq}{$:=$\,}
\newcommand{\continued}[1][4em]{\phantom{}\hspace{#1}}

% mess around with catcodes >:3
\catcode`\*=13
\catcode`\?=13
\catcode`\|=13
\catcode`\"=13
\catcode`\'=13
\catcode`\<=13
\catcode`\==13

\def*{\raisebox{0.25em}{$\char`\*$}}
\def\*{\char`\*}
\def?{$\char`\?$}
\def|{$\char`\|$}
\def"#1"{\terminal{#1}}
\def'#1'{\nonterminal{#1}}
\def<#1>{\textsc{#1}}
\def={\defeq}

}
    \end{tcolorbox}
    \caption{The pattern grammar understood by @CC}
    \label{lst:grammar}
\end{listing}

\begin{table}[htbp]
    \centering
    \begin{tblr}{c|c}
        \textbf{Keywords} & \textbf{Meaning} \\
        \hline
        \io{stitch}, \io{st} & stitch \\
        \io{mr}, \io{mc} & magic ring \\
        \io{chain}, \io{ch} & chain \\
        \io{slip}, \io{sl}, \io{slst}, \io{ss} & slip stitch \\
        \io{single}, \io{sc} & single crochet \\
        % \io{double}, \io{dc} & double crochet \\
        % \io{hdc} & half double crochet \\
        % \io{treble}, \io{triple}, \io{tr}, \io{tc} & treble crochet \\
        \io{increase}, \io{inc} & increase \\
        % \io{invisible}, \io{inv} & invisible \\
        \io{decrease}, \io{dec}, \io{sc2tog} & decrease \\
        \io{turn} & turn \\
        % \io{into}, \io{in} & into \\
        % \io{next} & next \\
        \io{repeat}, \io{rep} & repeat \\
        \io{from} & from \\
        \io{twice}, \io{two} & two times \\
        \io{thrice}, \io{three} & three times \\
        \io{times}, \io{time}, \io{x} & times \\
        \io{more} & more \\
    \end{tblr}
    \caption{The keywords and their meanings, as recognized by @CC.}
    \label{tbl:keywords}
\end{table}

The following terms are used throughout the manual:

\begin{description}
\item[@CC**] the name of the web application
\item[\term*{user}] the person who uses @CC, addressed by ``you''
\item[\term*{device}] whatever electronic device @CC is currently running on, inclusive of hardware and operating system

% Basic crochet things
\item[\term*{crochet}] the art of creating textiles by using a crochet hook and yarn
\item[\term*{yarn}] the material used in the art of crochet
\item[\term*{amigurumi}] the Japanese art of knitting or crocheting stuffed toys out of yarn
\item[\term*{pattern}] a set of textual instructions that detail how to create a crochet project
\item[\term*{stitch}] a simple, indivisible knot built on top of previous stitches, used to build patterns
\item[\term*{foundation}] the initial loop used to start a pattern

% CC things
\item[@layout**] the buttons, text boxes, widgets, and other elements visible on the screen in the @CC app
\item[@renarea**] the area of the screen used for displaying the @rm
\item[@rm**] the 3D object created and shown based on the inputted pattern text
\item[\term*{session}] an invocation of @CC
\item[@patbar**] the user interface widget to the left of the @renarea
\item[@patbox**] the text box within the @patbar which holds your input pattern text
\item[@asbs**] the set of buttons below the @patbox that allows the user to input stitches one at a time
\item[@nextcol**] the colour picker below the @asbs which determines the colour for the next added stitch
\item[@postpanel**] the user interface widget on the right side of the @renarea
\item[@stcol**] the button in the @postpanel which can change the colour for the selected stitch in the @rm
\item[@menu**] the user interface component above the @renarea, which contains the @ubtn, @dbtn, and @ebtn
\item[@ubtn**] the button in the @menu that, when pressed, allows you to select a text file from your local device such that the contents of that text file populate the @patbox
\item[@dbtn**] the button in the @menu that, when pressed, allows you to download your pattern text as a text file to your local device
\item[@ebtn**] the button in the @menu that, when pressed, allows you to download your @rm as an @OBJ file to your local device
\end{description}

\section{Other Abbreviations}

\begin{description}
\item[@ir**] any kind of data, not visible to the user, used internally by @CC to represent crochet patterns
\item[\Term*{GUI}] Graphical User Interface
\item[@OBJ**] a geometry definition file format, to which a @rm can be exported
\item[\emph{what you see is what you get} (\Term*{WYSIWYG})] software that displays content in a realistic form while it is being edited
\end{description}

\section{Basic User Interface Goals}

@CC aims to be a friendly \term{GUI} for crochet amateurs and enthusiasts.
Not much crochet modelling software exists, and most of this software does not provide fast and accessible rendering.
The main appeal of @CC is from the benefits of its \term{WYSIWYG} approach and its capability of fast pattern iteration.
Its basic goals are:

\begin{itemize}
\item It is simple for you to import your crochet pattern text into the @patbox, and it is simple for you to export a 3D object of the @rm in @OBJ format.
\item It is easy for you to view your @rm at many different angles.
\item It is easy for a pattern designer to iterate and make modifications to a pattern, to improve it.
\end{itemize}

\section{Organization of this Manual}

The remainder of this manual is organized primarily on use cases.
The \nameref{chp:use} chapter describes the basic use cases, including the possible GUI interactions in depth.
After the \nameref{chp:use} chapter, the \nameref{chp:troubleshooting} chapter describes what you should do if you encounter common errors as well as how to use @CC effectively.
Lastly, the \nameref{chp:limitations} chapter describes the restrictions on the current version of @CC.

\end{document}
