%!xelatex
\documentclass[main.tex]{subfiles}
\begin{document}

\chapter{Troubleshooting \& Tips}\label{chp:troubleshooting}

\section{Troubleshooting}

You input a pattern with an unrecognized stitch name.
\Solution{Modify the pattern to change or remove the unrecognized stitch name. The recognized stitches are listed in the \nameref{sec:terms} section.}

You input a pattern with invalid or unrecognized syntax.
\Solution{Modify the pattern to conform to the grammar laid out in the \nameref{sec:terms} section.}

You try to load a file that is not a crochet pattern.
\Solution{Open a plain text file instead, or open the desired file with the appropriate software and copy and paste the pattern text into the \CC{} application.}

You input a pattern with an excessively large amount of stitches.
\Solution{Reduce the number of stitches in the pattern. See the \nameref{chp:limitations} chapter for more details.}

You enter an invalid value in the settings panel. For example, a non-number when a number is required.
\Solution{Enter a valid number without spaces, thousands separators, or other symbols.}

You open the app in an outdated/incompatible browser.
\Solution{Use a device that satisfies the requirements listed in the \nameref{sec:csassumptions} section.}

You close the window without saving the pattern being typed.
\Solution{Ensure you have used the \ui{Download} function to save your work before closing the web browser window.}

You add more than one \term*{foundation} stitch.
\Solution{Remove the additional foundation stitches. \CC{} is only designed to render one crochet part at a time; different parts which are intended to be sewn together must be rendered separately.}

\section{Tips}

\begin{itemize}
\item Below are some sample patterns which you can copy-and-paste into the \PTI*, to help you get started:
    \begin{itemize}
    \item Disk: \\
        \io{%
        1. MR, sc 6 \\
        2. 6 (sc, inc) \\
        3. 6 (2 sc, inc) \\
        4. 6 (3 sc, inc) \\
        5. 6 (4 sc, inc) \\
        6. 6 (5 sc, inc)
        }

    \item Sphere: \\
        \io{%
        1. MR, sc 6 \\
        2. 6 (sc, inc) \\
        3. 6 (2 sc, inc) \\
        4. 6 (3 sc, inc) \\
        5. 6 (4 sc, inc) \\
        6-10. 36 sc \\
        11. 6 (4 sc, dec) \\
        12. 6 (3 sc, dec) \\
        13. 6 (2 sc, dec) \\
        14. 6 (sc, dec) \\
        15. 6 dec
        }
    \end{itemize}

\item Amigurumi patterns typically start with a magic circle, while clothing and similar textiles typically start with a slip knot and chain stitches.
\item Use the \ui{Download} function often to avoid losing your work.
\end{itemize}

\end{document}
