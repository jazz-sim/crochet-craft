%!xelatex
\documentclass[main.tex]{subfiles}
\begin{document}

\chapter{Use Cases}\label{chp:use}

The use cases of @CC are classified by the section of the GUI your initial interaction occurs in.

\section{Crochet Pattern Entry}

Crochet patterns may be entered\index{pattern!entry} in two ways: using the @patbox or the @asbs.

You may enter the pattern text in the @patbox*:
\begin{enumerate}
\item Any text may be typed into the @patbox, but only text that satisfies the grammar shall be considered valid.
\item If the resulting text within the @patbox is considered to be a valid pattern, the @renarea will update to display a new @rm corresponding to the updated pattern text. 
\end{enumerate}

You may consider alternative methods of direct text entry into the @patbox*:
\begin{itemize}
\item You may copy and paste text into the @patbox. This is done using the copy or paste shortcuts of the underlying device.
\item You may use speech-to-text to input the pattern text into the @patbox box.
\end{itemize}
Regardless of the method of text entry, the @renarea will update to display a new @rm corresponding to the updated pattern, if it is valid.

% In the case that the resultant text in the @patbox is invalid, the @renarea will not update, and will continue to display the same @rm corresponding to the last valid pattern text.

You may add one stitch to the end of a pattern using one of the @asbs*. Each @asb is labelled with the stitch that it will add to the @rm, and each button corresponds to a different stitch. 

\begin{enumerate}
\item You press the appropriate @asb. For example, to add a chain stitch, press the button labelled \ui{ch}.
\item The text corresponding to that stitch is added to the end of the @patbox.
\item If the resulting pattern text is valid, the @renarea will update to display a new @rm corresponding to the updated pattern text. 
\end{enumerate}

As an alternative to traditional text input, you may press any of the @asb multiple times to add multiple stitches. After each button press, the @patbox will update, and if the pattern text is still valid, the @renarea will update too.

In the case that the resultant text in the @patbox is invalid\index{pattern!invalid}, the @renarea will not update, and will continue to display the same @rm corresponding to the last valid pattern text.

\section{Viewing Crochet Pattern}

You may view the @rm* in the @renarea. Typically, no additional input is required to view the @rm; the @rm will be readily displayed on the screen under normal usage in the absence of your input.

You may interact with the @renarea using your mouse and keyboard to change the perspective that the @rm is viewed from.

Panning the camera:\index{camera!panning}
\begin{enumerate}
\item You may pan the camera by either (1) holding @rmc, or (2) holding \ui{Shift} and @lmc, and then moving your mouse in a direction within the @renarea.
\item The @renarea's perspective will shift in the direction that the mouse moves in.
\end{enumerate}

Rotating the camera:\index{camera!rotating}
\begin{enumerate}
\item You may rotate the camera by holding @lmc and moving your mouse in a direction.
\item The @renarea's perspective will rotate around the centre of the @rm along the direction in which the mouse moves.
\end{enumerate}

Zooming the camera:\index{camera!zooming}
\begin{enumerate}
\item You may zoom the camera in or out by scrolling the mouse wheel up or down. 
\item The @renarea's perspective will zoom in or out depending on the scroll direction. Scrolling down zooms the camera out, and scrolling up zooms the camera in.
\end{enumerate}

In the case that the @rm is no longer visible in the @renarea after the viewing perspective is changed,
@CC does not correct this. However, you can reset the camera to ensure the @rm is visible again.

Resetting the camera:\index{camera!resetting}
\begin{enumerate}
\item Click the @homebtn at the bottom-right corner of the @renarea.
\item The @renarea's perspective will be reset, nullifying any panning, rotating, and zooming done previously.
\end{enumerate}

\section{Interacting with the Rendered Model}\label{sec:interactrm}

You may interact with the @rm*\index{rendered model!interaction} using your mouse to either hover or select the individual stitches of the @rm.

Hovering:\index{rendered model!hovering}
\begin{enumerate}
\item You may hover over an individual stitch by moving the mouse pointer so it is over one of the stitches within the @rm.
\item The @renarea will update, so the stitch that is under the mouse pointer will glow.
\end{enumerate}

If the mouse pointer is over empty space, and there is no stitch under the pointer, no action is taken, and the @renarea will not change what is displayed.

Selecting:\index{rendered model!selecting}
\begin{enumerate}
\item First, a stitch in the @rm must be hovered over. 
\item You may then @lmc to select the hovered stitch. 
\end{enumerate}

In any use case involving a selected stitch\,---\,in particular, interactions with the @postpanel\,---\,the stitch involved in step~2 is the one that will be affected.

If a long press is performed, the hovered stitch will only be selected if the mouse pointer still lies on top of the stitch in the @rm when the @lmc is released.

If no stitch is being hovered over, but a @lmc is performed within the @renarea, @CC deselects the selected stitch, if there is one.

\section{Post-Processing the Rendered Model}

After a stitch has been selected, as described in the \nameref{sec:interactrm} section, the @postpanel* will be displayed. This can be used to alter a stitch's colour after rendering.

It is important to note that post-processing changes only affect the @rm and any exported 3D objects. The pattern text is not affected. This means, when the pattern text is changed, all changes made in the @postpanel are lost.

Changing the \term{colour} of a stitch:
\begin{enumerate}
\item After selecting the desired stitch, @lmc the @stcol.
\item In the resulting dialog, select the desired colour. The dialog may vary depending on your device's operating system or web browser.
\end{enumerate}

\section{Loading and Saving Crochet Patterns}

You can save your crochet pattern into a file on your device and load it later. This allows you to resume your work at a later time or share your work with others.

Saving your crochet pattern:\index{pattern!saving}
\begin{enumerate}
\item @Lmc the @dbtn* in the @menu*.
\item The current contents of the @patbox will be saved as a file onto your device. Depending on your device's operating system or web browser, the file may be saved in a standard location, or you may be prompted for a location to save the file.
\end{enumerate}

Loading a saved crochet pattern:\index{pattern!loading}
\begin{enumerate}
\item @Lmc the @ubtn* in the @menu*.
\item In the dialog box, select the file which was previously saved with the download function. The dialog box may vary depending on your device's operating system or web browser.
\end{enumerate}
In the case that the selected file is not a valid file that was previously saved with the download function, the error \io{File format not recognized} appears, and the file is not loaded.

You can export the rendered pattern as a 3D model file, which can be opened in third-party applications.

Exporting your crochet pattern:\index{pattern!exporting}
\begin{enumerate}
\item Once the @renarea displays the desired 3D model, @lmc the @ebtn* in the @menu*.
\item The 3D model currently displayed in the @renarea will be saved as a file onto your device in @OBJ* format. Depending on your device's operating system or web browser, the file may be saved in a standard location, or you may be prompted for a location to save the file. The file can be opened in a variety of third-party software that supports loading files in @OBJ format. Please consult the user manual for the third-party software to ensure it is capable of loading such files.
\end{enumerate}

\end{document}
